\section{Related work}
\label{sec:related_work}
In this paper, we investigated a total of eight papers. A study \cite{PilotStudy} by S. C. Jepsen et al. analyze the challenges for asset interoperability by conducting asset integration in the University's I4.0 laboratory, and conduct a pilot study to reveal that the maturity of assets interoperability readiness is at very different levels which need to be addressed. 

The authors in \cite{ResearchSetup} focus on enabling flexible production to reflect a production system. So that it can react dynamically and cost-effectively to the market needs. S. C. Jepsen et al. introduces a setup in the university's I4.0 laboratory that enables advanced and flexible experiments, in terms of use-cases for advanced production processes, and experiments with both robotics and automated production solution and software.

The authors from both \cite{analysisInterop} and \cite{AnalyticalModelInterOP} explain the crucial role played by the middleware in flexible I4.0 production, where S. C. Jepsen et al. \cite{analysisInterop} discusses a gap in knowledge of the interoperability between assets and I4.0 middleware based on previous literature and prior I4.0 lab experiences. The author in \cite{AnalyticalModelInterOP} analyzes the implications of multiple levels of interoperability in middleware software architecture.

H. Christense et al. \cite{agileArchitecting40} explores the limitation of Industry 4.0 automation for a small production and reports on early results from a project aiming at developing a software architecture fast, easy, and flexible for reconfiguration of a robotic manufacturing process using an agile and prototyping approach. S. C. Jepsen et al. \cite{ExperienceReport} argues that the Industry 4.0 vision is about efficient, flexible production supporting rapidly changing requirements through digitalization based enabling middleware software architecture. Here the authors present a proposal for a systematic approach to gather quality attribute requirements by presenting a three-phase process, and the application of it, in a collaboration project with participants from industry and academia, where the three phases are data collection, analyzing architectural requirements, and evaluating QAS.

The authors in \cite{ProductivityProductionSystem} propose a definition of reconfigurability for I4.0 middleware software architecture. They also demonstrate how the reconfigurability definition supports architectural reasoning about reconfigurable middleware through a case study on actual running middleware. S. C. Jepsen et al. in \cite{ReconfigureableIndustry} designs and evaluates a reconfigurable I4.0 middleware software architecture based on a developed reconfigurable quality attribute scenario (QAS).
