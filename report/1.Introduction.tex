\section{Introduction and Motivation}
%Describe I4.0 and challenges of I4.0 as a domain of architecure
%Describe the challenges of what a pharmaceutical company can face
%Describe what we are focusing on (QAs)

In this paper, a software architecture design is proposed in the Industry 4.0 domain. Industry 4.0 is a very complex domain, where almost all the parts are connected either directly or through the cloud. This brings many challenges when it comes to designing an architecture for this domain. It is essential that all the parts of the system are highly available and it is also essential that they are connected to each other. High availability is important, because the information exchange can only work effectively when all the systems are up and running close to 24/7, whereas the connection between the components provide the possibility for them to exchange information. This brought a new concept to architectures in the Industry 4.0 domain which is called middleware discussed in depth in \cite{analysisInterop}. The different systems in an Industry 4.0 architecture might have been written in different programming languages, and use completely different technologies, while all being responsible for different things. However, they must be highly interconnected and be highly available. The middleware in an Industry 4.0 is the key concept, which makes it easy to connect completely different systems ensuring that they can all communicate with each other. Based on those facts, quality attributes can be very important when designing architecture in an Industry 4.0 domain, to support use cases specific to their domain. The motivation of this paper is to propose an architecture that a pharmaceutical startup can use in their smart factory, which is a factory complying with the standards of Industry 4.0 described above. 


%For smart factories it is very important that every Industrial Internet of Things (IIoT) device should be connected in a way that they can share meaningful data with each other. On a software architectural level it means that different systems and subsystems should also be able to communicate with each other efficiently and without the loss of information. 